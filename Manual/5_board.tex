% !TeX root=main.tex
\chapter{コントローラボード}

本章では,コントローラボードの仕様とコントローラボードのマイコン向けの制御プログラムの
ビルド方法,制御プログラムの書き込みと動作確認の方法について説明します.
なお本章では,「FPGA リモコンボード V2」を V2 ボード,「FPGA リモコンボード V4」および
「SawareruBoard V1」を V4 ボードと呼称します.

%%%%%%%%%%%%%%%%%%%%%%%%%%%%%%%%%%%%%%%%%%%%%%%%%%%%%%%%%%%%%%%%%%%%%%%%%%%%%%
\section{ボードの基本仕様}

コントローラボードは,USB に対応した PIC18F マイコンを搭載しています.マイコンの
型番は,V2 ボードが PIC18F4450-I/P(40ピン DIP),V4 ボードが PIC18F45K50/I-PT
(44ピン QFP)です.マイコンの汎用 I/O ピンとして使用できる全てのピンが,ボードの
入出力または USB の D+/D- 信号に割当てられています.表\ref{table:pic_io}に,入出力
の割当ての一覧を示します.入出力(名前は DRFront における表記に準拠)のそれぞれに
対して,割当て先となるマイコンのピン番号とポート名を記載しています.

\begin{table}[ht]
 \centering
  \caption{コントローラボードの I/O のマイコンへの割当て.}
  \label{table:pic_io}
  \begin{tabular}{c|c|l|c|l||c|c|l|c|l} \hline
   入力 & \multicolumn{2}{c|}{V2ボード} & \multicolumn{2}{c||}{V4ボード} &
   出力 & \multicolumn{2}{c|}{V2ボード} & \multicolumn{2}{c}{V4ボード} \\ \hline \hline
   SW(0) & 25  & RC6 & 17 & RB7 & LD(0) &  9  & RE1 & 26 & RE1 \\
   SW(1) & 26  & RC7 & 16 & RB6 & LD(1) &  8  & RE0 & 25 & RE0 \\
   SW(2) & 33  & RB0 & 15 & RB5 & LD(2) &  7  & RA5 & 24 & RA5 \\
   SW(3) & 34  & RB1 & 14 & RB4 & LD(3) &  6  & RA4 & 23 & RA4 \\
   SW(4) & 35  & RB2 & 11 & RB3 & LD(4) &  5  & RA3 & 22 & RA3 \\
   SW(5) & 36  & RB3 & 10 & RB2 & LD(5) &  4  & RA2 & 21 & RA2 \\
   SW(6) & 37  & RB4 &  9 & RB1 & LD(6) &  3  & RA1 & 20 & RA1 \\
   SW(7) & 38  & RB5 &  8 & RB0 & LD(7) &  2  & RA0 & 19 & RA0 \\
   BTNR  & 39  & RB6 & 32 & RC0 & AN(0) & 17  & RC2 &  5 & RD7 \\
   BTNC  & 40  & RB7 & 30 & RA7 & AN(1) & 16  & RC1 &  4 & RD6 \\
   BTNL  &  1  & RE3 & 18 & RE3 & AN(2) & 15  & RC0 &  3 & RD5 \\
   BTND  & n/a & n/a & 27 & RE2 & AN(3) & 10  & RE2 &  2 & RD4 \\
   BTNU  & n/a & n/a & 31 & RA6 & CA    & 19  & RD0 & 41 & RD3 \\
         &     &     &    &     & CB    & 20  & RD1 & 39 & RD1 \\
         &     &     &    &     & CC    & 21  & RD2 & 44 & RC6 \\
         &     &     &    &     & CD    & 22  & RD3 & 38 & RD0 \\
         &     &     &    &     & CE    & 27  & RD4 & 36 & RC2 \\
         &     &     &    &     & CF    & 28  & RD5 &  1 & RC7 \\
         &     &     &    &     & CG    & 29  & RD6 & 40 & RD2 \\
         &     &     &    &     & DP    & 30  & RD7 & 35 & RC1 \\ \hline
 \end{tabular}
\end{table}

電源 5 V は CN1 の USB ポート(mini B または Type-C)USB バスパワーで供給されます.
電源が供給されている間は,ボード上の D1 LED が赤色で点灯します.マイコンへの制御
プログラムの書き込みは,CN2 の ICSP 端子を介して行います.

コントローラボードのすべての入力は負論理です.またそのうち,マイコンの RB ポート
および RE3 ポートに接続されている入力については,マイコンの内部プルアップを使用
します.それ以外の入力は,10 k$\Omega$ の抵抗で明示的にプルアップされます.
一方で,コントローラボードのすべての出力は正論理です.通常の LED は 1 k$\Omega$の,
7セグメント LED は 330 $\Omega$ の抵抗により電流が制限されます.

%%%%%%%%%%%%%%%%%%%%%%%%%%%%%%%%%%%%%%%
\section{制御プログラムの書き込み}

制御プログラムのソースコードおよびプロジェクト一式は,PIC/RemoCon\_v?.x に置かれて
います.プログラムは MicroChip 社 MPLAB X IDE v6.10,MPLAB Code Configurator (MCC)
v5.3.7,および XC8 v2.4.1 を用いてビルドできることを確認しています.

V4 ボード向けの制御プログラムをビルドするためには,あらかじめ MCC によるコード再生成
が必要です.PIC/RemoCon\_v4.x フォルダを MPLAB X IDE 上でプロジェクトとして開くと,
プロジェクトに必要なファイルが見つからない旨のエラーが表示されます.このエラーは一旦
無視して,「Window」→「MPLAB Code Configurator v5」→「MPLAB Code Configurator v5:
Open/Close」で MCC を開きます.MCC で画面左上の Project Resources から,Generate
ボタンをクリックします.これにより,必要なファイルが自動的に生成されます.

制御プログラムがビルドできたら,PICkit などの PIC ライタを CN2 に三角の印を合わせて
接続して,プログラムをマイコンに書き込んでください.

書き込んだプログラムを簡易的に動作確認するためには,Tera Term などの UART に対応した
ターミナルアプリから,直接コマンドを打ち込みます.以下の手順で動作確認を行います.

\begin{enumerate}
 \item PC とコントローラボードを USB で接続します.
 \item ターミナルアプリでコントローラボードのシリアルポートを開きます.デバイス名は
 通常「USB シリアル デバイス」となります.転送レートは 115,200 bps,データは 8 bit,
 ストップビットは 1 bit,パリティとフロー制御はいずれもなしに設定します.
 \item ターミナルアプリ上で「VX」と入力すると,画面上に「WY」と表示されます.
 \item コントローラボードの左端のスライドスイッチをオン,オフと操作します.スイッチを
 オンにしたときとオフにしたときに,画面上にそれぞれ「PU」「Pu」と表示されます.
 \item ターミナルアプリ上で「3g」「3H」と入力します.7セグメント LED の左端の桁
 が消灯したあと,その桁の小数点が点灯します.
 \item ターミナルアプリ上で「4G」と入力すると,左端から2番目の LED が点灯します.
\end{enumerate}

%%%%%%%%%%%%%%%%%%%%%%%%%%%%%%%%%%%%%%%%%%%%%%%%%%%%%%%%%%%%%%%%%%%%%%%%%%%%%%